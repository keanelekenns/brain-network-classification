\def\year{2022}\relax
\documentclass[letterpaper]{article}
% DO NOT CHANGE THIS
\usepackage{aaai22} % DO NOT CHANGE THIS
\usepackage{times} % DO NOT CHANGE THIS
\usepackage{helvet} % DO NOT CHANGE THIS
\usepackage{courier} % DO NOT CHANGE THIS
\usepackage[hyphens]{url} % DO NOT CHANGE THIS
\usepackage{graphicx} % DO NOT CHANGE THIS
\urlstyle{rm} % DO NOT CHANGE THIS
\def\UrlFont{\rm} % DO NOT CHANGE THIS
\usepackage{graphicx} % DO NOT CHANGE THIS
\usepackage{natbib} % DO NOT CHANGE THIS
\usepackage{caption} % DO NOT CHANGE THIS
\DeclareCaptionStyle{ruled}%
{labelfont=normalfont,labelsep=colon,strut=off}
\frenchspacing % DO NOT CHANGE THIS
\setlength{\pdfpagewidth}{8.5in} % DO NOT CHANGE THIS
\setlength{\pdfpageheight}{11in} % DO NOT CHANGE THIS
%
% PDF Info Is REQUIRED.
% For /Title, write your title in Mixed Case.
% Don’t use accents or commands. Retain the parentheses.
% For /Author, add all authors within the parentheses,
% separated by commas. No accents, special characters
% or commands are allowed.
% Keep the /TemplateVersion tag as is
\pdfinfo{
/Title (Explaining Models for Brain Network Classification)
/Author (Keanelek Enns, Tengkai Yu, Alex Thomo, Venkatesh Srinivasan)
/TemplateVersion (2022.1)
}

% Specify path to images
\graphicspath{ {./img/} }

% NOT SURE IF THESE PACKAGES ARE ALLOWED
\usepackage{amsmath}
\usepackage{amsfonts}

\author{
Keanelek Enns, Tengkai Yu, Alex Thomo, Venkatesh Srinivasan\\
}
\affiliations {
University of Victoria 3800 Finnerty Road Victoria, BC V8P 5C2\\
}

\title{Explaining Models for Brain Network Classification}

\begin{document}
\maketitle

\begin{abstract}
% The point of using Lorem ipsum is to roughly see where images and sections will land in the final result
LOREM IPSUM DOLOR SIT AMET, CONSECTETUR ADIPISCING ELIT. EXPRESSA VERO IN IIS AETATIBUS, QUAE IAM CONFIRMATAE SUNT. NOSTI, CREDO, ILLUD: NEMO PIUS EST, QUI PIETATEM-; ATQUE HAEC CONIUNCTIO CONFUSIOQUE VIRTUTUM TAMEN A PHILOSOPHIS RATIONE QUADAM DISTINGUITUR. VIDIT HOMERUS PROBARI FABULAM NON POSSE, SI CANTIUNCULIS TANTUS IRRETITUS VIR TENERETUR; QUAMQUAM TU HANC COPIOSIOREM ETIAM SOLES DICERE. LEGIMUS TAMEN DIOGENEM, ANTIPATRUM, MNESARCHUM, PANAETIUM, MULTOS ALIOS IN PRIMISQUE FAMILIAREM NOSTRUM POSIDONIUM. NEMO NOSTRUM ISTIUS GENERIS ASOTOS IUCUNDE PUTAT VIVERE. ET QUIDEM IURE FORTASSE, SED TAMEN NON GRAVISSIMUM EST TESTIMONIUM MULTITUDINIS. DUO REGES: CONSTRUCTIO INTERRETE. RATIO QUIDEM VESTRA SIC COGIT.

ERGO IN GUBERNANDO NIHIL, IN OFFICIO PLURIMUM INTEREST, QUO IN GENERE PECCETUR. QUIA NEC HONESTO QUIC QUAM HONESTIUS NEC TURPI TURPIUS. NON POTES ERGO ISTA TUERI, TORQUATE, MIHI CREDE, SI TE IPSE ET TUAS COGITATIONES ET STUDIA PERSPEXERIS; QUAE AUTEM NATURA SUAE PRIMAE INSTITUTIONIS OBLITA EST? QUID, SI REVIVISCANT PLATONIS ILLI ET DEINCEPS QUI EORUM AUDITORES FUERUNT, ET TECUM ITA LOQUANTUR?
\end{abstract}

\section{Introduction}

Graph classification, the process of labelling graphs as belonging to certain categories or classes, has applications in many important domains.
This paper will focus on brain network classification, a particularly interesting application of graph classification that can be used to provide insights and diagnoses for neurological disorders such as Autism Spectrum Disorder (ASD).
These insights and diagnoses can further our understanding of such disorders and can aid in providing treatment and support to improve the lives of individuals possessing them.

% This first sentence doesn't really belong here
AI has been shown to out perform humans significantly in a multitude of domains, and its utilization can provide great value \cite{grace2018,fogel2018,brzezicki2020,kahng2021}.
There are multiple ways to measure the quality of a graph classification model.
Metrics such as accuracy, precision, and recall are essential for evaluating any classifier \cite{elkan2012}, but recently there has been a trend towards explainability in the world of machine learning (ML) and artificial intelligence (AI) \cite{hassan2021,linardatos2020}.
This is because industries, governments, and organizations, especially those that deal with critical decision making such as the medical field, are hesitant to adopt prediction models without knowing \emph{how and why} they make decisions, regardless of how accurate these models are reported to be \cite{asan2020}.
Therefore, it is increasingly important to find a balance between classification models that are explainable and simple to understand, while also achieving high accuracy, precision, and recall scores.

Explainability in AI is not only about building confidence in its users.
Providing insights that may not have been detected through classical methods is another major benefit and may lead to further advancements in research.
These insights are especially helpful to neuroscientists when it comes to studying less understood neurological disorders.

% Either I should reword this to be less prescriptive, or we should include this in the discussion/conclusions instead.
ASD is a particularly difficult disorder to diagnose and the disorder itself has loose definitions (FIND A CITATION).
(We probably want to mention how it is a spectrum, which likely contributes to this).
If ML and AI models were to be used for diagnosing ASD in practice, biases of initial models could compound, as diagnoses from such models may inevitably be used as training data for future models.
For such a misunderstood disorder as ASD, this could potentially mutate the practical definition of the disorder or favour certain expressions of it.
Therefore it is critical that such models are highly explainable.
This allows experts to evaluate the biases of the models and provide correction.

In their paper, "Explainable Classification of Brain Networks via Contrast Subgraphs", Lanciano \emph{et al.} proposed a method for translating brain networks into two dimensional representations with a simple interpretation \cite{lanciano2020}.
This translation of a graph into a vector representation is known as a graph embedding and is fundamental to many graph classification problems \cite{goyal2018}.
The graph embedding employed by Lanciano \emph{et al.} involves the use of contrast subgraphs (CSs), which we define later in this paper.

% Reword this as it is outdated
In this study, we sought to replicate the findings of Lanciano \emph{et al.} as well as discover alternatives or improvements to the methods described in their paper \cite{lanciano2020}.
We were unable to replicate their results as much of the implementation was unavailable in the given replication package\footnote{Lanciano \emph{et al.}'s replication package can be found at \url{https://github.com/tlancian/contrast-subgraph} and was first accessed in January 2022.}.

In light of this, we summarize our contributions as follows:

\begin{itemize}
    \item We perform a replication of the novel work done by Lanciano \emph{et al.}
    \item We demonstrate various improvements to the CS method.
    \item We develop a new graph embedding technique named Discriminative Edges (DE) that provides improved accuracy and reduced computational time over the CS method, all while maintaining similar explainability.
    \item Taking advantage of the node identity aware (NIA) data, we explore deep learning solutions with neural networks and compare existing explanation strategies with the explanations of the more simplistic methods.
    \item We provide a replication package to recreate all the work done in this study, as well as provide a complete and optimized implementation of Lanciano \emph{et al.}'s graph embedding process.
\end{itemize}

The following section will discuss related works.
Preliminary information about the data sets and notation used for this study will be covered in the section that follows.
We will then describe the CS embedding technique as well as our own embedding technique DE.
The experiments conducted and the results obtained are outlined in the Experiments section, and finally, we end the paper with discussions and conclusions for this study and suggestions for future research.



\section{Related Work}

% Manuscripts page: http://fcon_1000.projects.nitrc.org/indi/abide/manuscripts.html

Talk about the shortcomings of the work from lanciano here \cite{lanciano2020}?

\cite{gutierrez2019}
\cite{tsourakakis2013}

\cite{thomas2020}
\cite{sewani2020}.

Should we mention \cite{perotti2022} and \cite{coupette2022}

SCOTT WILL NEED TO WRITE ABOUT THE NEURAL NETWORK RELATED WORKS. Look at references 25, 26, 38, 42 from Lanciano's paper. Specifically 26 (https://ieeexplore.ieee.org/document/8970823) which claims 99\% accuracy with DNN approach.

\section{Preliminaries}
% http://preprocessed-connectomes-project.org/abide/index.html
% Check to see if ABIDE II has also been similarly preprocessed
% https://www.frontiersin.org/10.3389/conf.fninf.2013.09.00041/event_abstract
\textbf{Data.}
The data sets in this paper originated from the Autism Brain Imaging Data Exchange (ABIDE) project \cite{craddock2013} and were processed by Lanciano \emph{et al.}
This processing resulted in groups of simple, undirected, and unweighted graphs (brain networks) defined over a common set of 116 vertices.
The brain networks are labelled with two classes: Typically Developed (TD) or Autism Spectrum Disorder (ASD).

The vertices of the brain networks correspond to regions of interest (ROIs) in the brain, thus we are studying a special case of graph classification where all considered graphs possess the same set of vertices.
This concept is called \emph{node identity awareness}.
The edges of each brain network correspond to functional (as opposed to structural) connections in the brain.

For a more thorough discussion of the data sets used and the processing that was performed by Lanciano \emph{et al.}, please see section 5 of the their paper \cite{lanciano2020}.

\textbf{Notation.}
We reuse the notation specified in Lanciano \emph{et al.}'s paper and reiterate the important notation here.

Let the $i^{th}$ brain network be represented as an undirected, unweighted graph $G_i = (V, E_i)$, where $V$ is the common vertex set representing the 116 ROIs of the brain as discussed previously (i.e. $|V| = 116$) and $E_i$ is the set of edges belonging to $G_i$ (note: $E_i \subset V \times V$).

Let a \emph{summary graph} corresponding to a set of brain networks $\mathcal{A}$ be a weighted, undirected graph $G^{\mathcal{A}} = (V, w^{\mathcal{A}})$, where $w^{\mathcal{A}}: V \times V \rightarrow \mathbb{R}_+$ is a weight function that assigns a value to each pair of vertices in $V$.
For vertices $u,v \in V$, we define $w^{\mathcal{A}}(u,v)$ to be the fraction of networks in $\mathcal{A}$ that contain the edge $(u,v)$.

\section{Contrast Subgraphs} \label{cs}

A contrast subgraph is defined as a subset of vertices that induces a dense subgraph in one graph and a sparse subgraph in another, assuming that the graphs share a common vertex set.
We echo the problem definition of finding contrast subgraphs from the work of Lanciano \emph{et al.}

\emph{Problem 1 (Contrast Subgraph). Given two sets of observation graphs, i.e. the condition group $\mathcal{A} = \{G^{\mathcal{A}}_1, . . . , G^{\mathcal{A}}_{|\mathcal{A}|}\}$ and the control group $\mathcal{B} = \{G^{\mathcal{B}}_1, . . . , G^{\mathcal{B}}_{|\mathcal{B}|}\}$, and corresponding summary graphs $G^{\mathcal{A}} = (V, w^{\mathcal{A}})$ and $G^{\mathcal{B}} = (V, w^{\mathcal{B}})$, we seek to find a subset of vertices $S^* \subseteq V$ that maximizes the contrast subgraph objective}
\begin{align*}
    \delta (S) = \sum_{u,v \in S} \left(w^{\mathcal{A}}(u,v) - w^{\mathcal{B}}(u,v) - \alpha\right)
\end{align*}
\emph{where $\alpha \in \mathbb{R}_+$ is a user-defined parameter.}

Their paper also defines a symmetric variant of the contrast subgraph problem (\emph{Problem 2}) which shares the same definition of \emph{Problem 1}, but with an alternative objective

\begin{align*}
    \sigma (S) = \sum_{u,v \in S} \left(|w^{\mathcal{A}}(u,v) - w^{\mathcal{B}}(u,v)| - \alpha\right)
\end{align*}

The parameter $\alpha$ is used to penalize large contrast subgraphs.
It can be adjusted to vary the proportion of edges that are considered detrimental to the contrast subgraph objective.

These problems can be simplified to finding a set of vertices that induce a dense subgraph in the difference of the two summary graphs.
Consider the difference network  $G^{\mathcal{A} - \mathcal{B}} = (V, w^{\mathcal{A} - \mathcal{B}})$, where \\ $w^{\mathcal{A} - \mathcal{B}}(u,v) = w^{\mathcal{A}}(u,v) - w^{\mathcal{B}}(u,v), \forall u,v \in V$.
In this context, finding a contrast subgraph is equivalent to finding a dense subgraph in the difference network.
There are many notions of density studied in the literature.
Lanciano \emph{et al.} base the constrast subgraph objective off of the optimal quasi-clique problem \cite{tsourakakis2013}.

Note that contrast subgraphs themselves do not constitute a graph embedding technique, but they can be used in the following ways to embed a graph.

\textbf{Problem 1.}
The original authors use contrast sugbraph overlap to create two features for each brain network.
Because the problem is asymmetric, two contrast subgraphs are needed.
One is found in the difference network $G^{TD - ASD}$ (which is the result of subtracting the summary graph $G^{TD}$ from the summary graph $G^{ASD}$) and the other is found in $G^{ASD - TD}$.
Each feature of a brain network's embedding corresponds to the number of edges in common between the brain network and each contrast subgraph.
% Maybe this image should be displayed in the experiments section, but alluded to here
Figure \ref{fig:prob1} positions some of the brain networks in two dimensions based on their embeddings for problem 1. 

\begin{figure}
    \centering
    \includegraphics[width=\columnwidth, keepaspectratio=true]{test.png}
    \caption{Show a representative plot for problem 1}
    \label{fig:prob1}
\end{figure}

\textbf{Problem 2.}
A single contrast subgraph is found in the difference network containing absolute valued edge weights (i.e. $|G^{TD - ASD}|$).
The contrast subgraph is used to induce subgraphs in both of the summary graphs (i.e. $G^{TD}$ and $G^{ASD}$) as well as each individual brain network.
The distances, computed as the $L_1$ norms, from the induced brain network to each summary graph are then used as the two features for this embedding.
% Maybe this image should be displayed in the experiments section, but alluded to here
Figure \ref{fig:prob2} positions some of the brain networks in two dimensions based on their embeddings for problem 2. 

\begin{figure}
    \centering
    \includegraphics[width=\columnwidth, keepaspectratio=true]{test.png}
    \caption{Show a representative plot for problem 2}
    \label{fig:prob2}
\end{figure}




\section{Improvements} \label{improvements}

Based on our experiments, the original contrast subgraph embedding techniques are quite slow, and their accuracies are not far above the baseline accuracy (i.e. the accuracy obtained when the most dominant class is always predicted).
Therefore we turned our attention to developing improvements for the contrast subgraph embedding technique.

\textbf{Quadratic Programming.}
We looked further into the work by Cadena \emph{et al.} which was repurposed by the original authors.
When creating a mathematical problem formulation for finding a dense subgraph in a signed network, Cadena \emph{et al.} proposed a quadratic programming problem.
In their study, they chose to approximate the quadratic program with a semidefinite program and a rounding technique \cite{cadena2016}.
Lanciano \emph{et al.} then used this implementation to find a contrast subgraph in their context.

However, because the goal is to find a dense subgraph in a small difference network (116 by 116), it seemed reasonable to implement a QP solver to solve the problem directly.
We used a Python library named CVXOPT\footnote{This library can be found at \url{https://cvxopt.org/}}.
None of the QP solver libraries we found were able to restrict potential solution vectors to discrete values such as \{-1,1\} (i.e. out or in), so we restricted solution values to [-1,1] rounding positive values to 1 and non-positive values to -1.

% I STILL NEED TO CALCULATE THIS!!!
The speed increase that resulted from changing the solver was significant as we will discuss in the results section.

\textbf{Local Search Optimization.}
The QP and SDP solvers only approximate the densest subgraph, so a local search step is used to improve the density of the chosen subgraph.
The localSearch algorithm developed by Tsourakakis \emph{et al.} and used by both Cadena \emph{et al.} and the original authors works by considering the density of the subgraph if each vertex outside the subgraph were to be added to the subgraph, adding them when it increases the density.
This is repeated until all the vertices have been considered.
Afterwards, the algorithm seeks to remove a single vertex from the subgraph and does so if it finds one that increases the density.
This process is then repeated until a maximum number of iterations have been reached, or no vertices can be added or taken away from the subgraph to increase its density.
The pseudocode for the original algorithm can be found in their paper \cite{tsourakakis2013}.

Considering the goal of the local search algorithm is to increase the density of the subgraph, we made a modification to the algorithm such that it attempts to remove as many vertices as it can in each iteration (removing them only if they increase the density).
This modified local search algorithm can be seen in the replication package for this study, and was found to increase the accuracy of the classifiers that used it over the original implementation.

\textbf{Top-k Contrast Subgraphs.}
Finally, we follow the original authors' suggestion for future work and use the top-k contrast subgraphs.
We did not implement it according to Balalau \emph{et al.} \cite{balalau2015} as suggested, but instead took the approach described by Tsourakakis \emph{et al.} in section 4.1 of their paper \cite{tsourakakis2013}.
We removed the vertices of each contrast subgraph as it was found and searched for a new contrast subgraph among the remaining vertices of the graph until all $k$ contrast subgraphs had been found.

The resulting $k$ embeddings for each new brain network were added together feature-wise to obtain the final embedding.



\section{Discriminative Edges}
In section 5.1 of Lanciano \emph{et al.}'s paper, it was shown that the degrees of each vertex in the studied brain networks did not vary significantly between the two classes \cite{lanciano2020}, yet the contrast subgraphs are defined as a set of vertices, and every edge induced by these vertices is used for classification.
In light of this, we thought it would be beneficial to consider only the \emph{edges} that are important for distinguishing between the classes.

The discriminative edges (DE) embedding technique employs a very simple computation for calculating the similarity of a new brain network with each of the two summary graphs.
Initially, the idea was to identify the $n$ most positive edges and the $n$ most negative edges in the difference network $G^{ASD - TD}$.
These are the edges with the greatest disparity between the two classes (positive edges are more common in ASD brain networks, and negative edges are more common in TD brain networks), and are thus, the most discriminative edges.

For each new brain network $G_i$, we calculate the dot product of the n most positive edges in $G^{ASD - TD}$ with the corresponding edges of $G_i$ (in this case, $G_i$ is unweighted, so present edges have a value of 1, and edges that are not present have a value of 0, however, this calculation would still be effective if $G_i$ were weighted).
This gives $G_i$'s similarity with the ASD class.
A similar calculation is done for the negative edges to give the similarity with the TD class.
These similarities are used as the two features of the technique.

With this incredibly simple algorithm, we achieved results that were very similar to those obtained from the contrast subgraph technique.
However, with further improvements, DE managed to surpass the performance of the contrast subgraph technique while remaining computationally efficient and easy to understand.

Rather than ignoring the values of edges in the difference network corresponding to edges that do not exist in $G_i$, we made use of this information by scaling the edge values of $G_i$ such that nonexistent edges have a value of -1 and present edges continue to have a value of 1 (this scaling can also be applied in the case where $G_i$ is weighted).
This means that not having an edge among the most discriminative edges actually counts against the similarity score for each case.

Finally, we added a third feature to the embedding technique.
This feature is derived in a very similar way, but uses every edge in the calculation, and thereby captures more information.
It is calculated by scaling the edge values of $G_i$ as discussed previously, multiplying each edge to its corresponding edge in the difference network, and summing the values.
This gives one score, a positive value implies similarity with the ASD class, whereas a negative value implies similarity with the TD class.
% Note that we may need to change the order I mentioned here depending on what the figures show in terms of positive and negative scores \ref{fig:DE}

The DE embeddings encode more information than the contrast subgraph embeddings, they are more efficient to compute, and they are still easy to visualize in three dimensions.



\section{Experiments}
LOREM IPSUM DOLOR SIT AMET, CONSECTETUR ADIPISCING ELIT. EXPRESSA VERO IN IIS AETATIBUS, QUAE IAM CONFIRMATAE SUNT. NOSTI, CREDO, ILLUD: NEMO PIUS EST, QUI PIETATEM-; ATQUE HAEC CONIUNCTIO CONFUSIOQUE VIRTUTUM TAMEN A PHILOSOPHIS RATIONE QUADAM DISTINGUITUR. VIDIT HOMERUS PROBARI FABULAM NON POSSE, SI CANTIUNCULIS TANTUS IRRETITUS VIR TENERETUR; QUAMQUAM TU HANC COPIOSIOREM ETIAM SOLES DICERE. LEGIMUS TAMEN DIOGENEM, ANTIPATRUM, MNESARCHUM, PANAETIUM, MULTOS ALIOS IN PRIMISQUE FAMILIAREM NOSTRUM POSIDONIUM. NEMO NOSTRUM ISTIUS GENERIS ASOTOS IUCUNDE PUTAT VIVERE. ET QUIDEM IURE FORTASSE, SED TAMEN NON GRAVISSIMUM EST TESTIMONIUM MULTITUDINIS. DUO REGES: CONSTRUCTIO INTERRETE. RATIO QUIDEM VESTRA SIC COGIT.

ERGO IN GUBERNANDO NIHIL, IN OFFICIO PLURIMUM INTEREST, QUO IN GENERE PECCETUR. QUIA NEC HONESTO QUIC QUAM HONESTIUS NEC TURPI TURPIUS. NON POTES ERGO ISTA TUERI, TORQUATE, MIHI CREDE, SI TE IPSE ET TUAS COGITATIONES ET STUDIA PERSPEXERIS; QUAE AUTEM NATURA SUAE PRIMAE INSTITUTIONIS OBLITA EST? QUID, SI REVIVISCANT PLATONIS ILLI ET DEINCEPS QUI EORUM AUDITORES FUERUNT, ET TECUM ITA LOQUANTUR?

LOREM IPSUM DOLOR SIT AMET, CONSECTETUR ADIPISCING ELIT. EXPRESSA VERO IN IIS AETATIBUS, QUAE IAM CONFIRMATAE SUNT. NOSTI, CREDO, ILLUD: NEMO PIUS EST, QUI PIETATEM-; ATQUE HAEC CONIUNCTIO CONFUSIOQUE VIRTUTUM TAMEN A PHILOSOPHIS RATIONE QUADAM DISTINGUITUR. VIDIT HOMERUS PROBARI FABULAM NON POSSE, SI CANTIUNCULIS TANTUS IRRETITUS VIR TENERETUR; QUAMQUAM TU HANC COPIOSIOREM ETIAM SOLES DICERE. LEGIMUS TAMEN DIOGENEM, ANTIPATRUM, MNESARCHUM, PANAETIUM, MULTOS ALIOS IN PRIMISQUE FAMILIAREM NOSTRUM POSIDONIUM. NEMO NOSTRUM ISTIUS GENERIS ASOTOS IUCUNDE PUTAT VIVERE. ET QUIDEM IURE FORTASSE, SED TAMEN NON GRAVISSIMUM EST TESTIMONIUM MULTITUDINIS. DUO REGES: CONSTRUCTIO INTERRETE. RATIO QUIDEM VESTRA SIC COGIT.

ERGO IN GUBERNANDO NIHIL, IN OFFICIO PLURIMUM INTEREST, QUO IN GENERE PECCETUR. QUIA NEC HONESTO QUIC QUAM HONESTIUS NEC TURPI TURPIUS. NON POTES ERGO ISTA TUERI, TORQUATE, MIHI CREDE, SI TE IPSE ET TUAS COGITATIONES ET STUDIA PERSPEXERIS; QUAE AUTEM NATURA SUAE PRIMAE INSTITUTIONIS OBLITA EST? QUID, SI REVIVISCANT PLATONIS ILLI ET DEINCEPS QUI EORUM AUDITORES FUERUNT, ET TECUM ITA LOQUANTUR?

LOREM IPSUM DOLOR SIT AMET, CONSECTETUR ADIPISCING ELIT. EXPRESSA VERO IN IIS AETATIBUS, QUAE IAM CONFIRMATAE SUNT. NOSTI, CREDO, ILLUD: NEMO PIUS EST, QUI PIETATEM-; ATQUE HAEC CONIUNCTIO CONFUSIOQUE VIRTUTUM TAMEN A PHILOSOPHIS RATIONE QUADAM DISTINGUITUR. VIDIT HOMERUS PROBARI FABULAM NON POSSE, SI CANTIUNCULIS TANTUS IRRETITUS VIR TENERETUR; QUAMQUAM TU HANC COPIOSIOREM ETIAM SOLES DICERE. LEGIMUS TAMEN DIOGENEM, ANTIPATRUM, MNESARCHUM, PANAETIUM, MULTOS ALIOS IN PRIMISQUE FAMILIAREM NOSTRUM POSIDONIUM. NEMO NOSTRUM ISTIUS GENERIS ASOTOS IUCUNDE PUTAT VIVERE. ET QUIDEM IURE FORTASSE, SED TAMEN NON GRAVISSIMUM EST TESTIMONIUM MULTITUDINIS. DUO REGES: CONSTRUCTIO INTERRETE. RATIO QUIDEM VESTRA SIC COGIT.

ERGO IN GUBERNANDO NIHIL, IN OFFICIO PLURIMUM INTEREST, QUO IN GENERE PECCETUR. QUIA NEC HONESTO QUIC QUAM HONESTIUS NEC TURPI TURPIUS. NON POTES ERGO ISTA TUERI, TORQUATE, MIHI CREDE, SI TE IPSE ET TUAS COGITATIONES ET STUDIA PERSPEXERIS; QUAE AUTEM NATURA SUAE PRIMAE INSTITUTIONIS OBLITA EST? QUID, SI REVIVISCANT PLATONIS ILLI ET DEINCEPS QUI EORUM AUDITORES FUERUNT, ET TECUM ITA LOQUANTUR?

LOREM IPSUM DOLOR SIT AMET, CONSECTETUR ADIPISCING ELIT. EXPRESSA VERO IN IIS AETATIBUS, QUAE IAM CONFIRMATAE SUNT. NOSTI, CREDO, ILLUD: NEMO PIUS EST, QUI PIETATEM-; ATQUE HAEC CONIUNCTIO CONFUSIOQUE VIRTUTUM TAMEN A PHILOSOPHIS RATIONE QUADAM DISTINGUITUR. VIDIT HOMERUS PROBARI FABULAM NON POSSE, SI CANTIUNCULIS TANTUS IRRETITUS VIR TENERETUR; QUAMQUAM TU HANC COPIOSIOREM ETIAM SOLES DICERE. LEGIMUS TAMEN DIOGENEM, ANTIPATRUM, MNESARCHUM, PANAETIUM, MULTOS ALIOS IN PRIMISQUE FAMILIAREM NOSTRUM POSIDONIUM. NEMO NOSTRUM ISTIUS GENERIS ASOTOS IUCUNDE PUTAT VIVERE. ET QUIDEM IURE FORTASSE, SED TAMEN NON GRAVISSIMUM EST TESTIMONIUM MULTITUDINIS. DUO REGES: CONSTRUCTIO INTERRETE. RATIO QUIDEM VESTRA SIC COGIT.

ERGO IN GUBERNANDO NIHIL, IN OFFICIO PLURIMUM INTEREST, QUO IN GENERE PECCETUR. QUIA NEC HONESTO QUIC QUAM HONESTIUS NEC TURPI TURPIUS. NON POTES ERGO ISTA TUERI, TORQUATE, MIHI CREDE, SI TE IPSE ET TUAS COGITATIONES ET STUDIA PERSPEXERIS; QUAE AUTEM NATURA SUAE PRIMAE INSTITUTIONIS OBLITA EST? QUID, SI REVIVISCANT PLATONIS ILLI ET DEINCEPS QUI EORUM AUDITORES FUERUNT, ET TECUM ITA LOQUANTUR?

LOREM IPSUM DOLOR SIT AMET, CONSECTETUR ADIPISCING ELIT. EXPRESSA VERO IN IIS AETATIBUS, QUAE IAM CONFIRMATAE SUNT. NOSTI, CREDO, ILLUD: NEMO PIUS EST, QUI PIETATEM-; ATQUE HAEC CONIUNCTIO CONFUSIOQUE VIRTUTUM TAMEN A PHILOSOPHIS RATIONE QUADAM DISTINGUITUR. VIDIT HOMERUS PROBARI FABULAM NON POSSE, SI CANTIUNCULIS TANTUS IRRETITUS VIR TENERETUR; QUAMQUAM TU HANC COPIOSIOREM ETIAM SOLES DICERE. LEGIMUS TAMEN DIOGENEM, ANTIPATRUM, MNESARCHUM, PANAETIUM, MULTOS ALIOS IN PRIMISQUE FAMILIAREM NOSTRUM POSIDONIUM. NEMO NOSTRUM ISTIUS GENERIS ASOTOS IUCUNDE PUTAT VIVERE. ET QUIDEM IURE FORTASSE, SED TAMEN NON GRAVISSIMUM EST TESTIMONIUM MULTITUDINIS. DUO REGES: CONSTRUCTIO INTERRETE. RATIO QUIDEM VESTRA SIC COGIT.

ERGO IN GUBERNANDO NIHIL, IN OFFICIO PLURIMUM INTEREST, QUO IN GENERE PECCETUR. QUIA NEC HONESTO QUIC QUAM HONESTIUS NEC TURPI TURPIUS. NON POTES ERGO ISTA TUERI, TORQUATE, MIHI CREDE, SI TE IPSE ET TUAS COGITATIONES ET STUDIA PERSPEXERIS; QUAE AUTEM NATURA SUAE PRIMAE INSTITUTIONIS OBLITA EST? QUID, SI REVIVISCANT PLATONIS ILLI ET DEINCEPS QUI EORUM AUDITORES FUERUNT, ET TECUM ITA LOQUANTUR?

LOREM IPSUM DOLOR SIT AMET, CONSECTETUR ADIPISCING ELIT. EXPRESSA VERO IN IIS AETATIBUS, QUAE IAM CONFIRMATAE SUNT. NOSTI, CREDO, ILLUD: NEMO PIUS EST, QUI PIETATEM-; ATQUE HAEC CONIUNCTIO CONFUSIOQUE VIRTUTUM TAMEN A PHILOSOPHIS RATIONE QUADAM DISTINGUITUR. VIDIT HOMERUS PROBARI FABULAM NON POSSE, SI CANTIUNCULIS TANTUS IRRETITUS VIR TENERETUR; QUAMQUAM TU HANC COPIOSIOREM ETIAM SOLES DICERE. LEGIMUS TAMEN DIOGENEM, ANTIPATRUM, MNESARCHUM, PANAETIUM, MULTOS ALIOS IN PRIMISQUE FAMILIAREM NOSTRUM POSIDONIUM. NEMO NOSTRUM ISTIUS GENERIS ASOTOS IUCUNDE PUTAT VIVERE. ET QUIDEM IURE FORTASSE, SED TAMEN NON GRAVISSIMUM EST TESTIMONIUM MULTITUDINIS. DUO REGES: CONSTRUCTIO INTERRETE. RATIO QUIDEM VESTRA SIC COGIT.

ERGO IN GUBERNANDO NIHIL, IN OFFICIO PLURIMUM INTEREST, QUO IN GENERE PECCETUR. QUIA NEC HONESTO QUIC QUAM HONESTIUS NEC TURPI TURPIUS. NON POTES ERGO ISTA TUERI, TORQUATE, MIHI CREDE, SI TE IPSE ET TUAS COGITATIONES ET STUDIA PERSPEXERIS; QUAE AUTEM NATURA SUAE PRIMAE INSTITUTIONIS OBLITA EST? QUID, SI REVIVISCANT PLATONIS ILLI ET DEINCEPS QUI EORUM AUDITORES FUERUNT, ET TECUM ITA LOQUANTUR?

LOREM IPSUM DOLOR SIT AMET, CONSECTETUR ADIPISCING ELIT. EXPRESSA VERO IN IIS AETATIBUS, QUAE IAM CONFIRMATAE SUNT. NOSTI, CREDO, ILLUD: NEMO PIUS EST, QUI PIETATEM-; ATQUE HAEC CONIUNCTIO CONFUSIOQUE VIRTUTUM TAMEN A PHILOSOPHIS RATIONE QUADAM DISTINGUITUR. VIDIT HOMERUS PROBARI FABULAM NON POSSE, SI CANTIUNCULIS TANTUS IRRETITUS VIR TENERETUR; QUAMQUAM TU HANC COPIOSIOREM ETIAM SOLES DICERE. LEGIMUS TAMEN DIOGENEM, ANTIPATRUM, MNESARCHUM, PANAETIUM, MULTOS ALIOS IN PRIMISQUE FAMILIAREM NOSTRUM POSIDONIUM. NEMO NOSTRUM ISTIUS GENERIS ASOTOS IUCUNDE PUTAT VIVERE. ET QUIDEM IURE FORTASSE, SED TAMEN NON GRAVISSIMUM EST TESTIMONIUM MULTITUDINIS. DUO REGES: CONSTRUCTIO INTERRETE. RATIO QUIDEM VESTRA SIC COGIT.

ERGO IN GUBERNANDO NIHIL, IN OFFICIO PLURIMUM INTEREST, QUO IN GENERE PECCETUR. QUIA NEC HONESTO QUIC QUAM HONESTIUS NEC TURPI TURPIUS. NON POTES ERGO ISTA TUERI, TORQUATE, MIHI CREDE, SI TE IPSE ET TUAS COGITATIONES ET STUDIA PERSPEXERIS; QUAE AUTEM NATURA SUAE PRIMAE INSTITUTIONIS OBLITA EST? QUID, SI REVIVISCANT PLATONIS ILLI ET DEINCEPS QUI EORUM AUDITORES FUERUNT, ET TECUM ITA LOQUANTUR?

LOREM IPSUM DOLOR SIT AMET, CONSECTETUR ADIPISCING ELIT. EXPRESSA VERO IN IIS AETATIBUS, QUAE IAM CONFIRMATAE SUNT. NOSTI, CREDO, ILLUD: NEMO PIUS EST, QUI PIETATEM-; ATQUE HAEC CONIUNCTIO CONFUSIOQUE VIRTUTUM TAMEN A PHILOSOPHIS RATIONE QUADAM DISTINGUITUR. VIDIT HOMERUS PROBARI FABULAM NON POSSE, SI CANTIUNCULIS TANTUS IRRETITUS VIR TENERETUR; QUAMQUAM TU HANC COPIOSIOREM ETIAM SOLES DICERE. LEGIMUS TAMEN DIOGENEM, ANTIPATRUM, MNESARCHUM, PANAETIUM, MULTOS ALIOS IN PRIMISQUE FAMILIAREM NOSTRUM POSIDONIUM. NEMO NOSTRUM ISTIUS GENERIS ASOTOS IUCUNDE PUTAT VIVERE. ET QUIDEM IURE FORTASSE, SED TAMEN NON GRAVISSIMUM EST TESTIMONIUM MULTITUDINIS. DUO REGES: CONSTRUCTIO INTERRETE. RATIO QUIDEM VESTRA SIC COGIT.

ERGO IN GUBERNANDO NIHIL, IN OFFICIO PLURIMUM INTEREST, QUO IN GENERE PECCETUR. QUIA NEC HONESTO QUIC QUAM HONESTIUS NEC TURPI TURPIUS. NON POTES ERGO ISTA TUERI, TORQUATE, MIHI CREDE, SI TE IPSE ET TUAS COGITATIONES ET STUDIA PERSPEXERIS; QUAE AUTEM NATURA SUAE PRIMAE INSTITUTIONIS OBLITA EST? QUID, SI REVIVISCANT PLATONIS ILLI ET DEINCEPS QUI EORUM AUDITORES FUERUNT, ET TECUM ITA LOQUANTUR?

LOREM IPSUM DOLOR SIT AMET, CONSECTETUR ADIPISCING ELIT. EXPRESSA VERO IN IIS AETATIBUS, QUAE IAM CONFIRMATAE SUNT. NOSTI, CREDO, ILLUD: NEMO PIUS EST, QUI PIETATEM-; ATQUE HAEC CONIUNCTIO CONFUSIOQUE VIRTUTUM TAMEN A PHILOSOPHIS RATIONE QUADAM DISTINGUITUR. VIDIT HOMERUS PROBARI FABULAM NON POSSE, SI CANTIUNCULIS TANTUS IRRETITUS VIR TENERETUR; QUAMQUAM TU HANC COPIOSIOREM ETIAM SOLES DICERE. LEGIMUS TAMEN DIOGENEM, ANTIPATRUM, MNESARCHUM, PANAETIUM, MULTOS ALIOS IN PRIMISQUE FAMILIAREM NOSTRUM POSIDONIUM. NEMO NOSTRUM ISTIUS GENERIS ASOTOS IUCUNDE PUTAT VIVERE. ET QUIDEM IURE FORTASSE, SED TAMEN NON GRAVISSIMUM EST TESTIMONIUM MULTITUDINIS. DUO REGES: CONSTRUCTIO INTERRETE. RATIO QUIDEM VESTRA SIC COGIT.

ERGO IN GUBERNANDO NIHIL, IN OFFICIO PLURIMUM INTEREST, QUO IN GENERE PECCETUR. QUIA NEC HONESTO QUIC QUAM HONESTIUS NEC TURPI TURPIUS. NON POTES ERGO ISTA TUERI, TORQUATE, MIHI CREDE, SI TE IPSE ET TUAS COGITATIONES ET STUDIA PERSPEXERIS; QUAE AUTEM NATURA SUAE PRIMAE INSTITUTIONIS OBLITA EST? QUID, SI REVIVISCANT PLATONIS ILLI ET DEINCEPS QUI EORUM AUDITORES FUERUNT, ET TECUM ITA LOQUANTUR?



\section{Discussion}
\textbf{Discussion of Node-Identity Awareness and its importance with respect to how we treat the data. We do not need to treat it strictly like a graph as lanciano et al do.}

Diagnoses do not just come from brain scans, they also come from behavioural records (VERIFY THAT STATEMENT).
If we wish to ensure the accuracy of AI models when providing diagnoses, it would make sense to include such behavioural information to train and build models.





LOREM IPSUM DOLOR SIT AMET, CONSECTETUR ADIPISCING ELIT. EXPRESSA VERO IN IIS AETATIBUS, QUAE IAM CONFIRMATAE SUNT. NOSTI, CREDO, ILLUD: NEMO PIUS EST, QUI PIETATEM-; ATQUE HAEC CONIUNCTIO CONFUSIOQUE VIRTUTUM TAMEN A PHILOSOPHIS RATIONE QUADAM DISTINGUITUR. VIDIT HOMERUS PROBARI FABULAM NON POSSE, SI CANTIUNCULIS TANTUS IRRETITUS VIR TENERETUR; QUAMQUAM TU HANC COPIOSIOREM ETIAM SOLES DICERE. LEGIMUS TAMEN DIOGENEM, ANTIPATRUM, MNESARCHUM, PANAETIUM, MULTOS ALIOS IN PRIMISQUE FAMILIAREM NOSTRUM POSIDONIUM. NEMO NOSTRUM ISTIUS GENERIS ASOTOS IUCUNDE PUTAT VIVERE. ET QUIDEM IURE FORTASSE, SED TAMEN NON GRAVISSIMUM EST TESTIMONIUM MULTITUDINIS. DUO REGES: CONSTRUCTIO INTERRETE. RATIO QUIDEM VESTRA SIC COGIT.

ERGO IN GUBERNANDO NIHIL, IN OFFICIO PLURIMUM INTEREST, QUO IN GENERE PECCETUR. QUIA NEC HONESTO QUIC QUAM HONESTIUS NEC TURPI TURPIUS. NON POTES ERGO ISTA TUERI, TORQUATE, MIHI CREDE, SI TE IPSE ET TUAS COGITATIONES ET STUDIA PERSPEXERIS; QUAE AUTEM NATURA SUAE PRIMAE INSTITUTIONIS OBLITA EST? QUID, SI REVIVISCANT PLATONIS ILLI ET DEINCEPS QUI EORUM AUDITORES FUERUNT, ET TECUM ITA LOQUANTUR?
LOREM IPSUM DOLOR SIT AMET, CONSECTETUR ADIPISCING ELIT. EXPRESSA VERO IN IIS AETATIBUS, QUAE IAM CONFIRMATAE SUNT. NOSTI, CREDO, ILLUD: NEMO PIUS EST, QUI PIETATEM-; ATQUE HAEC CONIUNCTIO CONFUSIOQUE VIRTUTUM TAMEN A PHILOSOPHIS RATIONE QUADAM DISTINGUITUR. VIDIT HOMERUS PROBARI FABULAM NON POSSE, SI CANTIUNCULIS TANTUS IRRETITUS VIR TENERETUR; QUAMQUAM TU HANC COPIOSIOREM ETIAM SOLES DICERE. LEGIMUS TAMEN DIOGENEM, ANTIPATRUM, MNESARCHUM, PANAETIUM, MULTOS ALIOS IN PRIMISQUE FAMILIAREM NOSTRUM POSIDONIUM. NEMO NOSTRUM ISTIUS GENERIS ASOTOS IUCUNDE PUTAT VIVERE. ET QUIDEM IURE FORTASSE, SED TAMEN NON GRAVISSIMUM EST TESTIMONIUM MULTITUDINIS. DUO REGES: CONSTRUCTIO INTERRETE. RATIO QUIDEM VESTRA SIC COGIT.

ERGO IN GUBERNANDO NIHIL, IN OFFICIO PLURIMUM INTEREST, QUO IN GENERE PECCETUR. QUIA NEC HONESTO QUIC QUAM HONESTIUS NEC TURPI TURPIUS. NON POTES ERGO ISTA TUERI, TORQUATE, MIHI CREDE, SI TE IPSE ET TUAS COGITATIONES ET STUDIA PERSPEXERIS; QUAE AUTEM NATURA SUAE PRIMAE INSTITUTIONIS OBLITA EST? QUID, SI REVIVISCANT PLATONIS ILLI ET DEINCEPS QUI EORUM AUDITORES FUERUNT, ET TECUM ITA LOQUANTUR?
LOREM IPSUM DOLOR SIT AMET, CONSECTETUR ADIPISCING ELIT. EXPRESSA VERO IN IIS AETATIBUS, QUAE IAM CONFIRMATAE SUNT. NOSTI, CREDO, ILLUD: NEMO PIUS EST, QUI PIETATEM-; ATQUE HAEC CONIUNCTIO CONFUSIOQUE VIRTUTUM TAMEN A PHILOSOPHIS RATIONE QUADAM DISTINGUITUR. VIDIT HOMERUS PROBARI FABULAM NON POSSE, SI CANTIUNCULIS TANTUS IRRETITUS VIR TENERETUR; QUAMQUAM TU HANC COPIOSIOREM ETIAM SOLES DICERE. LEGIMUS TAMEN DIOGENEM, ANTIPATRUM, MNESARCHUM, PANAETIUM, MULTOS ALIOS IN PRIMISQUE FAMILIAREM NOSTRUM POSIDONIUM. NEMO NOSTRUM ISTIUS GENERIS ASOTOS IUCUNDE PUTAT VIVERE. ET QUIDEM IURE FORTASSE, SED TAMEN NON GRAVISSIMUM EST TESTIMONIUM MULTITUDINIS. DUO REGES: CONSTRUCTIO INTERRETE. RATIO QUIDEM VESTRA SIC COGIT.

ERGO IN GUBERNANDO NIHIL, IN OFFICIO PLURIMUM INTEREST, QUO IN GENERE PECCETUR. QUIA NEC HONESTO QUIC QUAM HONESTIUS NEC TURPI TURPIUS. NON POTES ERGO ISTA TUERI, TORQUATE, MIHI CREDE, SI TE IPSE ET TUAS COGITATIONES ET STUDIA PERSPEXERIS; QUAE AUTEM NATURA SUAE PRIMAE INSTITUTIONIS OBLITA EST? QUID, SI REVIVISCANT PLATONIS ILLI ET DEINCEPS QUI EORUM AUDITORES FUERUNT, ET TECUM ITA LOQUANTUR?


\section{Conclusion}
LOREM IPSUM DOLOR SIT AMET, CONSECTETUR ADIPISCING ELIT. EXPRESSA VERO IN IIS AETATIBUS, QUAE IAM CONFIRMATAE SUNT. NOSTI, CREDO, ILLUD: NEMO PIUS EST, QUI PIETATEM-; ATQUE HAEC CONIUNCTIO CONFUSIOQUE VIRTUTUM TAMEN A PHILOSOPHIS RATIONE QUADAM DISTINGUITUR. VIDIT HOMERUS PROBARI FABULAM NON POSSE, SI CANTIUNCULIS TANTUS IRRETITUS VIR TENERETUR; QUAMQUAM TU HANC COPIOSIOREM ETIAM SOLES DICERE. LEGIMUS TAMEN DIOGENEM, ANTIPATRUM, MNESARCHUM, PANAETIUM, MULTOS ALIOS IN PRIMISQUE FAMILIAREM NOSTRUM POSIDONIUM. NEMO NOSTRUM ISTIUS GENERIS ASOTOS IUCUNDE PUTAT VIVERE. ET QUIDEM IURE FORTASSE, SED TAMEN NON GRAVISSIMUM EST TESTIMONIUM MULTITUDINIS. DUO REGES: CONSTRUCTIO INTERRETE. RATIO QUIDEM VESTRA SIC COGIT.

ERGO IN GUBERNANDO NIHIL, IN OFFICIO PLURIMUM INTEREST, QUO IN GENERE PECCETUR. QUIA NEC HONESTO QUIC QUAM HONESTIUS NEC TURPI TURPIUS. NON POTES ERGO ISTA TUERI, TORQUATE, MIHI CREDE, SI TE IPSE ET TUAS COGITATIONES ET STUDIA PERSPEXERIS; QUAE AUTEM NATURA SUAE PRIMAE INSTITUTIONIS OBLITA EST? QUID, SI REVIVISCANT PLATONIS ILLI ET DEINCEPS QUI EORUM AUDITORES FUERUNT, ET TECUM ITA LOQUANTUR?


\subsection{Future Work}
\begin{itemize}
    \item Consider spaciality of ROIs.
    \item Consider different preprocessing of ABIDE dataset.
    \item Run experiments with ABIDE II (if we don't get to it)
\end{itemize}


ERGO IN GUBERNANDO NIHIL, IN OFFICIO PLURIMUM INTEREST, QUO IN GENERE PECCETUR. QUIA NEC HONESTO QUIC QUAM HONESTIUS NEC TURPI TURPIUS. NON POTES ERGO ISTA TUERI, TORQUATE, MIHI CREDE, SI TE IPSE ET TUAS COGITATIONES ET STUDIA PERSPEXERIS; QUAE AUTEM NATURA SUAE PRIMAE INSTITUTIONIS OBLITA EST? QUID, SI REVIVISCANT PLATONIS ILLI ET DEINCEPS QUI EORUM AUDITORES FUERUNT, ET TECUM ITA LOQUANTUR?
LOREM IPSUM DOLOR SIT AMET, CONSECTETUR ADIPISCING ELIT. EXPRESSA VERO IN IIS AETATIBUS, QUAE IAM CONFIRMATAE SUNT. NOSTI, CREDO, ILLUD: NEMO PIUS EST, QUI PIETATEM-; ATQUE HAEC CONIUNCTIO CONFUSIOQUE VIRTUTUM TAMEN A PHILOSOPHIS RATIONE QUADAM DISTINGUITUR. VIDIT HOMERUS PROBARI FABULAM NON POSSE, SI CANTIUNCULIS TANTUS IRRETITUS VIR TENERETUR; QUAMQUAM TU HANC COPIOSIOREM ETIAM SOLES DICERE. LEGIMUS TAMEN DIOGENEM, ANTIPATRUM, MNESARCHUM, PANAETIUM, MULTOS ALIOS IN PRIMISQUE FAMILIAREM NOSTRUM POSIDONIUM. NEMO NOSTRUM ISTIUS GENERIS ASOTOS IUCUNDE PUTAT VIVERE. ET QUIDEM IURE FORTASSE, SED TAMEN NON GRAVISSIMUM EST TESTIMONIUM MULTITUDINIS. DUO REGES: CONSTRUCTIO INTERRETE. RATIO QUIDEM VESTRA SIC COGIT.

\appendix

% References and End of Paper
% These lines must be placed at the end of your paper
\bibliography{refs}
\end{document}
