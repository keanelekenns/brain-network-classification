\def\year{2022}\relax
\documentclass[letterpaper]{article}
% DO NOT CHANGE THIS
\usepackage{aaai22} % DO NOT CHANGE THIS
\usepackage{times} % DO NOT CHANGE THIS
\usepackage{helvet} % DO NOT CHANGE THIS
\usepackage{courier} % DO NOT CHANGE THIS
\usepackage[hyphens]{url} % DO NOT CHANGE THIS
\usepackage{graphicx} % DO NOT CHANGE THIS
\urlstyle{rm} % DO NOT CHANGE THIS
\def\UrlFont{\rm} % DO NOT CHANGE THIS
\usepackage{graphicx} % DO NOT CHANGE THIS
\usepackage{natbib} % DO NOT CHANGE THIS
\usepackage{caption} % DO NOT CHANGE THIS
\DeclareCaptionStyle{ruled}%
{labelfont=normalfont,labelsep=colon,strut=off}
\frenchspacing % DO NOT CHANGE THIS
\setlength{\pdfpagewidth}{8.5in} % DO NOT CHANGE THIS
\setlength{\pdfpageheight}{11in} % DO NOT CHANGE THIS
%
% PDF Info Is REQUIRED.
% For /Title, write your title in Mixed Case.
% Don’t use accents or commands. Retain the parentheses.
% For /Author, add all authors within the parentheses,
% separated by commas. No accents, special characters
% or commands are allowed.
% Keep the /TemplateVersion tag as is
\pdfinfo{
/Title (Explaining Models for Brain Network Classification)
/Author (Keanelek Enns, Tengkai Yu, Alex Thomo, Venkatesh Srinivasan)
/TemplateVersion (2022.1)
}

\author{
Keanelek Enns, Tengkai Yu, Alex Thomo, Venkatesh Srinivasan\\
}
\affiliations {
University of Victoria 3800 Finnerty Road Victoria, BC V8P 5C2\\
}

\title{Explaining Models for Brain Network Classification}

\begin{document}
\maketitle
\begin{abstract}
TODO
\end{abstract}

\section{Introduction}

There are a number of different areas we could focus on:
\begin{enumerate}
    \item Autism Spectrum Disorder and its impact
    \item Model Explainability and Interpretability (explainable models vs explanation techniques for black box models)
\end{enumerate}


Contributions:
\begin{enumerate}
    \item Replication of Lanciano et al \cite{lanciano2020} + replication package
    \item Improvements to Lanciano et al's technique + discriminative edges technique + graph embedding evaluation module
    \item Using black box models on dataset (DNN and GNN)
    \item Investigate explanation techniques for black box models
\end{enumerate}
  
\section{Related Work}
\textbf{I think we need to spend a lot of time here. Looking at the manuscripts page, I am overwhelmed and feel that we have not adequately made ourselves aware of the existing work in the field. I do not feel confident saying our work is novel...}

Manuscripts page: http://fcon\_1000.projects.nitrc.org/indi/abide/manuscripts.html

Talk about the shortcomings of the work from lanciano here \cite{lanciano2020}?

\section{Problem Statement}
Not sure exactly where we are focusing our problem. Here are some options:
\begin{enumerate}
    \item Creating explainable models (node identity aware data)
    \item Exploring explanation techniques for black box models (node identity aware data) 
    \item Brain network classification
\end{enumerate}

\subsection{Data}
\textbf{We need to do more research on the dataset and consider using the ABIDE II dataset instead of just ABIDE I. It looks like only ABIDE I has been preprocessed from the link below, but perhaps someone has preprocessed ABIDE II in a similar fashion.}

http://preprocessed-connectomes-project.org/abide/index.html

https://www.frontiersin.org/10.3389/conf.fninf.2013.09.00041/event\_abstract



\textbf{Discussion of Node-Identity Awareness and its importance with respect to how we treat the data. We do not need to treat it strictly like a graph as lanciano et al do.}

\section{Experiments}

\section{Discussion}

\section{Conclusion}

\subsection{Future Work}
\begin{itemize}
    \item Consider spaciality of ROIs.
    \item Consider different preprocessing of ABIDE dataset.
    \item Run experiments with ABIDE II (if we don't get to it)
\end{itemize}

\appendix

% References and End of Paper
% These lines must be placed at the end of your paper
\bibliography{refs}
\end{document}
