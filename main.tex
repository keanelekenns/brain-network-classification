\documentclass[sigconf]{acmart}
% % Packages
\usepackage{soul}
% \usepackage{cite}
% \usepackage{balance}
% \usepackage{listings}
% \usepackage{amsmath,amssymb,amsfonts}
% \usepackage{algorithmic}
% \usepackage{graphicx}
% \usepackage{textcomp}
% \usepackage{xcolor}
% \usepackage{booktabs}
% \usepackage{enumitem}
% \usepackage{hyperref}
% \usepackage{totpages}
% \usepackage{subcaption}

%% \BibTeX command to typeset BibTeX logo in the docs
\AtBeginDocument{%
  \providecommand\BibTeX{{%
    \normalfont B\kern-0.5em{\scshape i\kern-0.25em b}\kern-0.8em\TeX}}}

\begin{document}

% Specify path to images
\graphicspath{ {./img/} }


%%
%% The "title" command has an optional parameter,
%% allowing the author to define a "short title" to be used in page headers.
\title[Contrast Subgraphs of Brain Networks]{Replication of "Explainable Classification of Brain Networks via Contrast Subgraphs"}

\author{Keanelek Enns}
\email{keanelekenns@uvic.ca}
\affiliation{%
  \institution{University of Victoria}
  \department{Computer Science}
  \streetaddress{PO Box 1700 STN CSC}
  \city{Victoria}
  \state{British Columbia}
  \country{Canada}
  \postcode{V8W 2Y2}
}


%%
%% By default, the full list of authors will be used in the page
%% headers. Often, this list is too long, and will overlap
%% other information printed in the page headers. This command allows
%% the author to define a more concise list
%% of authors' names for this purpose.
\renewcommand{\shortauthors}{K. Enns}

%%
%% The abstract is a short summary of the work to be presented in the
%% article.
\begin{abstract}
\hl{TODO}
\end{abstract}

%%
%% The code below is generated by the tool at http://dl.acm.org/ccs.cfm.
%%
\begin{CCSXML}
<ccs2012>
   <concept>
       <concept_id>10002950.10003624.10003633.10010917</concept_id>
       <concept_desc>Mathematics of computing~Graph algorithms</concept_desc>
       <concept_significance>500</concept_significance>
       </concept>
   <concept>
       <concept_id>10010405.10010444.10010446</concept_id>
       <concept_desc>Applied computing~Consumer health</concept_desc>
       <concept_significance>300</concept_significance>
       </concept>
 </ccs2012>
\end{CCSXML}

\ccsdesc[500]{Mathematics of computing~Graph algorithms}
\ccsdesc[300]{Applied computing~Consumer health}

%%
%% Keywords. The author(s) should pick words that accurately describe
%% the work being presented. Separate the keywords with commas.
\keywords{graph analytics, graph algorithms, brain networks, densest subgraph}

\settopmatter{printfolios=true} % for page numbering

%% This command processes the author and affiliation and title
%% information and builds the first part of the formatted document.
\maketitle

\section{Introduction} \label{intro}

\cite{lanciano2020}
\cite{goldberg1984}

\section{Conclusion} \label{conclusion}


\bibliographystyle{ACM-Reference-Format}
\bibliography{refs.bib}

\appendix
\section*{APPENDICES}
\section{Artifacts} \label{artifacts}

Project Repository:
\url{https://github.com/keanelekenns/contrast-subgraph}

\end{document}
\endinput